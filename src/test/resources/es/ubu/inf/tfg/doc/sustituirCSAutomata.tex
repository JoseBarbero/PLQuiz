\documentclass[11pt,a4paper]{report}
\usepackage[spanish]{babel}
\selectlanguage{spanish}
\usepackage[utf8]{inputenc}
\usepackage{graphicx}
\begin{document}
\paragraph{}

\paragraph{}
\textbf{1.-} Completa la tabla de función de transición para el AFD que se obtendría de aplicar el método de construcción de subconjuntos al AFND de la figura
\begin{figure}[ht!]
\centering
\includegraphics[width=90mm]{.jpg}
\end{figure}

\begin{tabular} {| c | c |c |c | l |}
\hline 
& a & b & c & \\ \hline
(A) & B & C & D & 0 1 2 4 5 7 8 9 11 13 \\ \hline
B & E & F & D & 3 8 9 10 11 \\ \hline
C & E & C & D & 5 6 7 8 9 11 \\ \hline
(D) & B & C & D & 1 2 4 5 7 8 9 11 12 13 \\ \hline
E & E & F & D & 9 10 11 \\ \hline
F & F & F & F & \\ \hline
\end{tabular}\paragraph{}
\textbf{2.-} Completa la tabla de función de transición para el AFD que se obtendría de aplicar el método de construcción de subconjuntos al AFND de la figura
\begin{figure}[ht!]
\centering
\includegraphics[width=90mm]{.jpg}
\end{figure}

\begin{tabular} {| c | c |c | l |}
\hline 
& a & b & \\ \hline
A & B & C & 0 1 2 4 7 \\ \hline
B & D & E & 1 2 3 4 6 7 8 9 11 \\ \hline
C & B & C & 1 2 4 5 6 7 \\ \hline
D & F & G & 1 2 3 4 6 7 8 9 10 11 13 14 16 \\ \hline
E & H & I & 1 2 4 5 6 7 12 13 14 16 \\ \hline
(F) & F & G & 1 2 3 4 6 7 8 9 10 11 13 14 15 16 18 \\ \hline
(G) & H & I & 1 2 4 5 6 7 12 13 14 16 17 18 \\ \hline
(H) & D & E & 1 2 3 4 6 7 8 9 11 15 18 \\ \hline
(I) & B & C & 1 2 4 5 6 7 17 18 \\ \hline
\end{tabular}

\end{document}