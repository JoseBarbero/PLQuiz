\documentclass[11pt,a4paper]{report}
\usepackage[spanish]{babel}
\selectlanguage{spanish}
\usepackage[utf8]{inputenc}
\usepackage{graphicx}
\begin{document}
\paragraph{}

\paragraph{}
\textbf{1.-} Completa el árbol sintáctico siguiendo el algoritmo de Aho-Sethi-Ullman para la expresión regular $ ((a|b^*)a^*c)^* $
\begin{figure}[ht!]
\centering
\includegraphics[width=90mm]{}
\end{figure}

\begin{tabular} {| c | c | c | c |}\hline
 & anulable? & primera-pos & última-pos\\ \hline
A & No & 1, 2, 3, 4, 5 & 5\\ \hline
B & Si & 1, 2, 3, 4 & 4\\ \hline
C & No & 5 & 5\\ \hline
D & No & 1, 2, 3, 4 & 4\\ \hline
E & Si & 1, 2, 3 & 1, 2, 3\\ \hline
F & No & 4 & 4\\ \hline
G & Si & 1, 2 & 1, 2\\ \hline
H & Si & 3 & 3\\ \hline
I & No & 1 & 1\\ \hline
J & Si & 2 & 2\\ \hline
K & No & 3 & 3\\ \hline
L & No & 2 & 2\\ \hline
\end{tabular}\paragraph{}
\textbf{2.-} Completa el árbol sintáctico siguiendo el algoritmo de Aho-Sethi-Ullman para la expresión regular $ (a|b^*)c^*a $
\begin{figure}[ht!]
\centering
\includegraphics[width=90mm]{}
\end{figure}

\begin{tabular} {| c | c | c | c |}\hline
 & anulable? & primera-pos & última-pos\\ \hline
A & No & 1, 2, 3, 4 & 5\\ \hline
B & No & 1, 2, 3, 4 & 4\\ \hline
C & No & 5 & 5\\ \hline
D & Si & 1, 2, 3 & 1, 2, 3\\ \hline
E & No & 4 & 4\\ \hline
F & Si & 1, 2 & 1, 2\\ \hline
G & Si & 3 & 3\\ \hline
H & No & 1 & 1\\ \hline
I & Si & 2 & 2\\ \hline
J & No & 3 & 3\\ \hline
K & No & 2 & 2\\ \hline
\end{tabular}\paragraph{}
\textbf{3.-} Completa el árbol sintáctico siguiendo el algoritmo de Aho-Sethi-Ullman para la expresión regular $ (a|b)^*a(a|b)(a|b) $
\begin{figure}[ht!]
\centering
\includegraphics[width=90mm]{}
\end{figure}

\begin{tabular} {| c | c | c | c |}\hline
 & anulable? & primera-pos & última-pos\\ \hline
A & No & 1, 2, 3 & 8\\ \hline
B & No & 1, 2, 3 & 6, 7\\ \hline
C & No & 8 & 8\\ \hline
D & No & 1, 2, 3 & 4, 5\\ \hline
E & No & 6, 7 & 6, 7\\ \hline
F & No & 1, 2, 3 & 3\\ \hline
G & No & 4, 5 & 4, 5\\ \hline
H & No & 6 & 6\\ \hline
I & No & 7 & 7\\ \hline
J & Si & 1, 2 & 1, 2\\ \hline
K & No & 3 & 3\\ \hline
L & No & 4 & 4\\ \hline
M & No & 5 & 5\\ \hline
N & No & 1, 2 & 1, 2\\ \hline
O & No & 1 & 1\\ \hline
P & No & 2 & 2\\ \hline
\end{tabular}

\end{document}