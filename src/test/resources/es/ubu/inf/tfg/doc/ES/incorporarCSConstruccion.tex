% !TeX program = xelatex
\documentclass[11pt,a4paper,table,answers]{exam} % <== Answers are printed
%\documentclass[11pt,a4paper,table]{exam} % <== Answers are NOT printed

\usepackage[spanish]{babel}
\selectlanguage{spanish}
\usepackage[utf8]{inputenc}
\usepackage{amssymb}

\usepackage{graphicx}
\usepackage{adjustbox} % Some graphs were too wide

\usepackage{xcolor} % table option is passed as documentclass option
\usepackage{booktabs}
\colorlet{azul}{blue!40!black}

\newcommand{\h}[1]{\ifprintanswers\textcolor{azul}{\bf#1}\else{\phantom{\bf#1}}\fi}

\newcommand{\myincludegraphics}[1]{%
\begin{center}
\begin{adjustbox}{max size={\textwidth}{\textheight}}
    \includegraphics{#1}
\end{adjustbox}
\end{center}
} % /myincludegraphics

\newlength{\graphicheigth}
\newsavebox{\graphicbox}
\newcommand{\myincludegraphicssol}[1]{%
    \sbox{\graphicbox}{%
        \begin{adjustbox}{max size={\textwidth}{\textheight}}
            \includegraphics{#1}
        \end{adjustbox}
    }
    \settoheight{\graphicheigth}{\usebox{\graphicbox}}
    \addtolength{\graphicheigth}{4ex} % increase a litle bit the heigth
    \ifprintanswers
        \begin{center}
            \usebox{\graphicbox}
        \end{center} 
    \else
        \makeemptybox{\graphicheigth}
    \fi
}

\begin{document}



\paragraph{1.-}\label{p1}
Siguiendo el método de McNaughton-Yamada-Thompson, dibuja el autómata generado por la expresión regular:
\[
    ((a|b^*)a^*c)^*
\]
    \myincludegraphicssol{}

\paragraph{2.-}\label{p2}
Siguiendo el método de McNaughton-Yamada-Thompson, dibuja el autómata generado por la expresión regular:
\[
    (a|b^*)c^*a
\]
    \myincludegraphicssol{}

\paragraph{3.-}\label{p3}
Siguiendo el método de McNaughton-Yamada-Thompson, dibuja el autómata generado por la expresión regular:
\[
    (a|b)^*a(a|b)(a|b)
\]
    \myincludegraphicssol{}

\end{document}
