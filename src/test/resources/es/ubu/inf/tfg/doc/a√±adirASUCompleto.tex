\documentclass[11pt,a4paper]{report}
\usepackage[spanish]{babel}
\selectlanguage{spanish}
\usepackage[utf8]{inputenc}
\usepackage{graphicx}
\begin{document}
\paragraph{}

\paragraph{}
\textbf{1.-} Aplicar el algoritmo de Aho-Sethi-Ullman a la expresión regular: \textbf{((a\textbar b*)a*c)*} \\
\\
Expresión aumentada: ((((a\textbar b*)·a*)·c)*·\$) \\

\begin{tabular} {| c | l |}
\hline
n & stePos(n) \\ \hline
1 & 3, 4 \\ \hline
2 & 2, 3, 4 \\ \hline
3 & 3, 4 \\ \hline
4 & 1, 2, 3, 4, 5 \\ \hline
5 & - \\ \hline
\end{tabular}
\quad
\begin{tabular} {| c | c |c |c | l |}
\hline 
& a & b & c & \\ \hline
(A) & B & C & A & 1 2 3 4 5 \\ \hline
B & B & D & A & 3 4 \\ \hline
C & B & C & A & 2 3 4 \\ \hline
D & D & D & D & \\ \hline
\end{tabular}\paragraph{}
\textbf{2.-} Aplicar el algoritmo de Aho-Sethi-Ullman a la expresión regular: \textbf{(a\textbar b*)c*a} \\
\\
Expresión aumentada: ((((a\textbar b*)·c*)·a)·\$) \\

\begin{tabular} {| c | l |}
\hline
n & stePos(n) \\ \hline
1 & 3, 4 \\ \hline
2 & 2, 3, 4 \\ \hline
3 & 3, 4 \\ \hline
4 & 5 \\ \hline
5 & - \\ \hline
\end{tabular}
\quad
\begin{tabular} {| c | c |c |c | l |}
\hline 
& a & b & c & \\ \hline
A & B & C & D & 1 2 3 4 \\ \hline
(B) & E & F & D & 3 4 5 \\ \hline
C & E & C & D & 2 3 4 \\ \hline
D & E & F & D & 3 4 \\ \hline
(E) & F & F & F & 5 \\ \hline
F & F & F & F & \\ \hline
\end{tabular}\paragraph{}
\textbf{3.-} Aplicar el algoritmo de Aho-Sethi-Ullman a la expresión regular: \textbf{(a\textbar b)*a(a\textbar b)(a\textbar b)} \\
\\
Expresión aumentada: (((((a\textbar b)*·a)·(a\textbar b))·(a\textbar b))·\$) \\

\begin{tabular} {| c | l |}
\hline
n & stePos(n) \\ \hline
1 & 1, 2, 3 \\ \hline
2 & 1, 2, 3 \\ \hline
3 & 4, 5 \\ \hline
4 & 6, 7 \\ \hline
5 & 6, 7 \\ \hline
6 & 8 \\ \hline
7 & 8 \\ \hline
8 & - \\ \hline
\end{tabular}
\quad
\begin{tabular} {| c | c |c | l |}
\hline 
& a & b & \\ \hline
A & B & A & 1 2 3 \\ \hline
B & C & D & 1 2 3 4 5 \\ \hline
C & E & F & 1 2 3 4 5 6 7 \\ \hline
D & G & H & 1 2 3 6 7 \\ \hline
(E) & E & F & 1 2 3 4 5 6 7 8 \\ \hline
(F) & G & H & 1 2 3 6 7 8 \\ \hline
(G) & C & D & 1 2 3 4 5 8 \\ \hline
(H) & B & A & 1 2 3 8 \\ \hline
\end{tabular}

\end{document}