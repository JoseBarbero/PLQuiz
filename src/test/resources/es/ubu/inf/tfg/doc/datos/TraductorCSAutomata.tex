\paragraph{}
\textbf{{numero}.-} Completa la tabla de función de transición para el AFD que se obtendría de aplicar el método de construcción de subconjuntos al AFND de la figura
\begin{figure}[ht!]
\centering
\includegraphics[width=90mm]{}
\end{figure}

\begin{tabular} {| c | c |c |c | l |}
\hline 
& a & b & c & \\ \hline
(A) & B & C & D & 0-2, 4, 5, 7-9, 11, 13\\ \hline
B & E & F & D & 3, 8-11\\ \hline
C & E & C & D & 5-9, 11\\ \hline
(D) & B & C & D & 1, 2, 4, 5, 7-9, 11-13\\ \hline
E & E & F & D & 9-11\\ \hline
F & F & F & F & $\varnothing$\\ \hline
\end{tabular}
