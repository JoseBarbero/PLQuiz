\paragraph{}
\textbf{<%0%>.-} Completa la tabla de función de transición para el AFD que se obtendría de aplicar el método de construcción de subconjuntos al AFND de la figura
\begin{figure}[ht!]
\centering
\includegraphics[width=90mm]{.jpg}
\end{figure}

\begin{tabular} {| c | c |c |c | l |}
\hline 
& a & b & c & \\ \hline
(A) & B & C & D & 0 1 2 4 5 7 8 9 11 13 \\ \hline
B & E & F & D & 3 8 9 10 11 \\ \hline
C & E & C & D & 5 6 7 8 9 11 \\ \hline
(D) & B & C & D & 1 2 4 5 7 8 9 11 12 13 \\ \hline
E & E & F & D & 9 10 11 \\ \hline
F & F & F & F & \\ \hline
\end{tabular}