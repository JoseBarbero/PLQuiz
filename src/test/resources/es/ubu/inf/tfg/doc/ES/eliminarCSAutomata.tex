\documentclass[11pt,a4paper,table,answers]{exam} % <== Answers are printed
%\documentclass[11pt,a4paper,table]{exam} % <== Answers are NOT printed

\usepackage[spanish]{babel}
\selectlanguage{spanish}
\usepackage[utf8]{inputenc}
\usepackage{amssymb}

\usepackage{graphicx}
\usepackage{adjustbox} % Some graphs were too wide

\usepackage{xcolor} % table option is passed as documentclass option
\usepackage{booktabs}
\colorlet{azul}{blue!40!black}

\newcommand{\h}[1]{\ifprintanswers\textcolor{azul}{\bf#1}\else{\phantom{\bf#1}}\fi}

\newcommand{\myincludegraphics}[1]{%
\begin{center}
\begin{adjustbox}{max size={\textwidth}{\textheight}}
    \includegraphics{#1}
\end{adjustbox}
\end{center}
} % /myincludegraphics

\newlength{\graphicheigth}
\newsavebox{\graphicbox}
\newcommand{\myincludegraphicssol}[1]{%
    \sbox{\graphicbox}{%
        \begin{adjustbox}{max size={\textwidth}{\textheight}}
            \includegraphics{#1}
        \end{adjustbox}
    }
    \settoheight{\graphicheigth}{\usebox{\graphicbox}}
    \addtolength{\graphicheigth}{4ex} % increase a litle bit the heigth
    \ifprintanswers
        \begin{center}
            \usebox{\graphicbox}
        \end{center} 
    \else
        \makeemptybox{\graphicheigth}
    \fi
}

\begin{document}



\paragraph{1.-}\label{p1}
Completa la tabla de función de transición para el AFD que se obtendría de aplicar el método de construcción de subconjuntos al AFND de la figura
    \myincludegraphics{}
\rowcolors{2}{gray!25}{white}
\begin{tabular} {c@{\hspace{4mm}}ccc@{\hspace{4mm}}l}
\toprule % =============================
$\mathcal{Q}$ & a & b & c & \emph{Estados del AFND}\\
\midrule %-------------------------------
\h{(A)} & \h{B} & \h{C} & \h{D} & \h{0-2, 4, 5, 7-9, 11, 13}\\
\h{B} & \h{E} & \h{F} & \h{D} & \h{3, 8-11}\\
\h{C} & \h{E} & \h{C} & \h{D} & \h{5-9, 11}\\
\h{(D)} & \h{B} & \h{C} & \h{D} & \h{1, 2, 4, 5, 7-9, 11-13}\\
\h{E} & \h{E} & \h{F} & \h{D} & \h{9-11}\\
\h{F} & \h{F} & \h{F} & \h{F} & \h{$\varnothing$}\\
\bottomrule % =============================
\end{tabular}

\paragraph{2.-}\label{p2}
Completa la tabla de función de transición para el AFD que se obtendría de aplicar el método de construcción de subconjuntos al AFND de la figura
    \myincludegraphics{}
\rowcolors{2}{gray!25}{white}
\begin{tabular} {c@{\hspace{4mm}}ccc@{\hspace{4mm}}l}
\toprule % =============================
$\mathcal{Q}$ & a & b & c & \emph{Estados del AFND}\\
\midrule %-------------------------------
\h{A} & \h{B} & \h{C} & \h{D} & \h{0, 1, 3, 4, 6-8, 10}\\
\h{(B)} & \h{E} & \h{F} & \h{D} & \h{2, 7, 8, 10, 11}\\
\h{C} & \h{E} & \h{C} & \h{D} & \h{4-8, 10}\\
\h{D} & \h{E} & \h{F} & \h{D} & \h{8-10}\\
\h{(E)} & \h{F} & \h{F} & \h{F} & \h{11}\\
\h{F} & \h{F} & \h{F} & \h{F} & \h{$\varnothing$}\\
\bottomrule % =============================
\end{tabular}

\end{document}
