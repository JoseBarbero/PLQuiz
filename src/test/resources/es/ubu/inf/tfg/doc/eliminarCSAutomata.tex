\documentclass[11pt,a4paper]{report}
\usepackage[spanish]{babel}
\selectlanguage{spanish}
\usepackage[utf8]{inputenc}
\usepackage{graphicx}
\begin{document}
\paragraph{}

\paragraph{}
\textbf{1.-} Completa la tabla de función de transición para el AFD que se obtendría de aplicar el método de construcción de subconjuntos al AFND de la figura
\begin{figure}[ht!]
\centering
\includegraphics[width=90mm]{}
\end{figure}

\begin{tabular} {| c | c |c |c | l |}
\hline 
& a & b & c & \\ \hline
(A) & B & C & D & 0-2, 4, 5, 7-9, 11, 13\\ \hline
B & E & F & D & 3, 8-11\\ \hline
C & E & C & D & 5-9, 11\\ \hline
(D) & B & C & D & 1, 2, 4, 5, 7-9, 11-13\\ \hline
E & E & F & D & 9-11\\ \hline
F & F & F & F & $\varnothing$\\ \hline
\end{tabular}\paragraph{}
\textbf{2.-} Completa la tabla de función de transición para el AFD que se obtendría de aplicar el método de construcción de subconjuntos al AFND de la figura
\begin{figure}[ht!]
\centering
\includegraphics[width=90mm]{}
\end{figure}

\begin{tabular} {| c | c |c |c | l |}
\hline 
& a & b & c & \\ \hline
A & B & C & D & 0, 1, 3, 4, 6-8, 10\\ \hline
(B) & E & F & D & 2, 7, 8, 10, 11\\ \hline
C & E & C & D & 4-8, 10\\ \hline
D & E & F & D & 8-10\\ \hline
(E) & F & F & F & 11\\ \hline
F & F & F & F & $\varnothing$\\ \hline
\end{tabular}

\end{document}