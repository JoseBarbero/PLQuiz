\documentclass[a4paper,11pt,oneside]{memoir}

% Castellano
\usepackage[spanish]{babel}
\selectlanguage{spanish}
\usepackage[utf8]{inputenc}
\usepackage{placeins}

% Imagenes
\usepackage{graphicx}
\newcommand{\imagen}[2]{
	\begin{figure}[!h]
		\centering
		\includegraphics[width=0.9\textwidth]{#1}
		\caption{#2}
	\end{figure}
	\FloatBarrier
}

\graphicspath{ {./img/} }

% Capítulos
\chapterstyle{bianchi}
\newcommand{\capitulo}[2]{
	\setcounter{chapter}{#1}
	\setcounter{section}{0}
	\chapter*{#2}
	\addcontentsline{toc}{chapter}{#2}
	\markboth{#2}{#2}
}

% Apéndices
\renewcommand{\appendixname}{Apéndice}
\renewcommand*\cftappendixname{\appendixname}

\newcommand{\apendice}[1]{
	\renewcommand{\thechapter}{A}
	\chapter{#1}
	
}

% Formato de portada
\makeatletter
\def\maketitle{
  \null
  \thispagestyle{empty}
  \vfill
  \begin{center}\leavevmode
    \normalfont
    {\LARGE \@author\par}
    \hrulefill\par
    {\huge \@title\par}
    \vskip 1cm
  \end{center}
  \vfill
  \null
  \cleardoublepage
  }
\makeatother

% Datos de portada
\title{Trabajo Fin de Grado: PLQuiz}
\author{Roberto Izquierdo Amo}
\date{}

\begin{document}

\maketitle

\frontmatter

% Abstract en castellano
\renewcommand*\abstractname{Resumen}
\begin{abstract}
Desarrollo de una herramienta de escritorio que ayude a generar preguntas de test aleatorias (tipo quiz, cloze, de texto libre...) sobre problemas de algoritmos de análisis léxico.
El formato utilizado para generar las preguntas será importable a entornos virtuales de aprendizaje (Moodle), y permitirá su impresión en papel a través de la obtención de código \LaTeX .
\end{abstract}

\renewcommand*\abstractname{Descriptores}
\begin{abstract}
Generación de cuestiones, análisis léxico, expresiones regulares, autómatas finitos.
\end{abstract}

\clearpage

% Abstract en inglés
\renewcommand*\abstractname{Abstract}
\begin{abstract}
Development of a desktop application that generates random test questions (quiz, cloze, free text...) about lexical analysis.
The format used to produce the questions should be compatible with virtual learning frameworks (such as Moodle), and allow for physical printing by generating \LaTeX\ code.
\end{abstract}

\renewcommand*\abstractname{Keywords}
\begin{abstract}
Question generation, lexical analysis, regular expresions, finite automata.
\end{abstract}

\clearpage

% Indices
\tableofcontents

\clearpage

\listoffigures

\clearpage

\listoftables

\clearpage

\mainmatter

%\capitulo{1}{Objetivos del proyecto}

\section{Objetivos técnicos}
Los objetivos del proyecto, vistos desde el punto de vista del desarrollo técnico de la aplicación, pueden desglosarse en los siguientes puntos:
\begin{itemize}
	\item Desarrollo de una interfaz gráfica multiplataforma.
	\item Procesado y manipulación de expresiones regulares mediante el uso de un procesador de textos, mediante el uso de la herramienta \emph{JavaCC}.
	\item Resolución de problemas de aplicación del algoritmo de Aho-Sethi-Ullman a partir de las expresiones regulares procesadas.
	\item Resolución de problemas de aplicación del algoritmo de McNaughton-Yamada-Thompson a partir de las expresiones regulares procesadas.
	\item Agregación de los problemas en cuestionarios.
	\item Exportado de cuestionarios a dos formatos:
	\begin{itemize}
		\item Formato \LaTeX{}, generando cuestionarios listos para impresión, con intención de ser usados como examenes en papel.
		\item Formato \emph{XML} compatible con la plataforma de aprendizaje virtual Moodle.
	\end{itemize}
	\item Elaboración de un documento detallando los aspectos relevantes del desarrollo y funcionamiento de la aplicación.
\end{itemize}

Sin embargo los objetivos no han permanecido estáticos durante el periodo de desarrollo.
A los puntos iniciales se han añadido los siguientes:
\begin{itemize}
	\item Generación de problemas aleatorios, utilizando algoritmos de programación genética y métodos de búsqueda.
	\item Generación de problemas en bloque, produciendo cuestionarios completos con una variedad de problemas distintos.
	\item Exportado de imágenes en múltiples formatos, incluyendo exportado directo, exportado de programas \emph{Graphviz} y codificación en base 64.
	\item Exportado de cuestionarios a un formato adicional, HTML, permitiendo la vista previa de cuestionarios dentro de la interfaz gráfica.
	\item Implantado de un sistema de `log', facilitando el diagnóstico de problemas durante el desarrollo de la aplicación.
\end{itemize}

\section{Objetivos académicos}
Además de los objetivos relacionados directamente con el desarrollo del proyecto concreto, se consideran otros más generales, particularmente los detallados en la guía docente.
De estos destaco especialmente los siguientes:
\begin{itemize}
	\item Aplicación de los conocimientos teóricos y prácticos adquiridos en al asignatura al desarrollo de un proyecto de más envergadura y durante un periodo de tiempo considerable.
	\item Planteamiento y resolución de un problema real, con aplicaciones y usos concretos.
	\item Aprendizaje de nuevos temas relacionados con la titulación, trabajando con nuevos métodos y tecnologías y adaptando a ellos el modo de trabajo.
	\item Desarrollo de la capacidad de exposición y comunicación, especialmente durante la defensa, en el desarrollo de la memoria y en la preparación de la documentación de la aplicación.
\end{itemize}

\section{Objetivos personales}
En último lugar, la realización de este proyecto me ha dado la oportunidad de completar una serie de objetivos a nivel personal.
Destaco especialmente el aprendizaje de nuevas tecnologías y herramientas, que aunque ya me resultaban conocidas o familiares, no había tenido oportunidad de aplicar seriamente.

Entre las herramientas utilizadas en el proyecto, las utilizadas para el `logging', SLF4J y Logback, resultan mucho más útiles al ser aplicadas a bases de código grandes.
Su uso en el proyecto no solo ha facilitado el diagnóstico de errores y el control del flujo del programa, sino que me ha proporcionado conocimientos que probablemente pueda aplicar más adelante.

También destaco el uso de la versión 8 de Java, que apareció cuando el proyecto ya estaba avanzado.
Especialmente el añadido de los `stream', que permiten el manejo de datos de manera similar a la programación funcional, resultó de ayuda durante el desarrollo de ciertas partes de la aplicación.
Su uso en el proyecto me ha permitido ponerme al día de las novedades en Java, utilizándolas para resolver problemas reales.
%\chapter*{Conceptos teóricos}
\capitulo{3}{Técnicas y herramientas}

\section{Técnicas}
Las técnicas empleadas en la aplicación y su desarrollo son las siguientes:

\subsection{SCRUM}
SCRUM\footnote{https://www.scrum.org/} es una metodología ágil que se define como
\begin{quote}
Un marco de trabajo por el cual las personas pueden acometer problemas complejos 
adaptativos, a la vez que entregar productos del máximo valor posible productiva y 
creativamente\cite{scrum}.
\end{quote}
Esta metodología se centra en crear procesos iterativos e incrementales, basandose en experiencias basadas para realizar estimaciones de rendimientos futuros, en base a las cuales tomar decisiones y controlar riesgos.

\subsection{UML}
UML\footnote{http://www.uml.org/} (\emph{Unified Modeling Language}) es un lenguaje de modelado utilizado en la estructura, comportamiento y arquitectura de aplicaciones.
El modelado de aplicaciones se considera una parte fundamental del diseño de las mismas, permitiendo examinar las funcionalidades desde un elevado nivel de abstracción.

\section{Herramientas}
Las herramientas utilizadas durante el desarrollo del proyecto son las siguientes:

\subsection{Java}
Java es un lenguaje de programación de propósito general \cite{jls8}, concurrente, basado en clases y orientado a objetos.
Se trata de un lenguaje fuerte y estáticamente tipado, y distingue claramente entre errores en tiempo de compilación y errores en tiempo de ejecución.

La compilación de un programa Java implica su traducción a representación `byte code', que es independiente de la plataforma.
Esta representación intermedia pasa a ejecutarse sobre la máquina virtual Java.
Por lo tanto, el código Java sigue la filosofía `write once, run anywhere', siendo compatible con cualquier arquitectura que disponga de una máquina virtual compatible.

Java es un lenguaje de relativo alto nivel, en el sentido de que los detalles de representación no son accesibles al programador.
Incluye una administración automática del almacenamiento de datos en forma del recolector de basura, diseñado para evitar los problemas relacionados con la liberación manual de memoria.

El lenguaje Java es, a fecha de 2014, uno de los lenguajes de programación más populares del mundo \cite{website:tiobe}.

\subsubsection{Java 8}
Java SE 8 es la edición más reciente del lenguaje de programación Java, y representa la mayor evolución del mismo en su historia.

Añade a la especificación un conjunto relativamente pequeño de nuevas características, mediante las cuales permite combinar modelos de programación orientada a objetos y funcional.
Estos cambios pretenden favorecer una serie de buenas prácticas --- inmutabilidad, ausencia de estado, composición --- a la vez que mantienen las características de Java --- legibilidad, simplicidad y universalidad.

Las librerías de la plataforma Java mantienen una evolución paralela a la del lenguaje.
Esto significa que el uso de las nuevas características --- expresiones lambda, referencias a métodos, interfaces funcionales --- es directamente compatible.
\cite{jls8}

\subsubsection{Eclipse}
El Eclipse Software Development Kit (Eclipse SDK)\footnote{http://eclipse.org/} es tanto un entorno de desarrollo integrado (IDE) para Java como una base para productos basados en Eclipse Platform.

Eclipse aparece como una herramienta propietaria de IBM que pretendía unificar los entornos de desarrollo ofrecidos a sus clientes y posibilitar la reutilización de componentes entre los mismos.
El proyecto de código abierto surge en 2001, con la fundación del Eclipse Consortium, que se encarga del desarrollo y administración del proyecto en la actualidad. \cite{website:eclipseFAQ}

Uno de los puntos fuertes de Eclipse es su capacidad para integrar una variedad de componentes mediante su sistema de plugins, lo cual permite trabajar con una variedad de lenguajes y herramientas.

\subsection{JUnit 4}
JUnit es un sistema software utilizado para realizar pruebas sobre código Java\footnote{http://junit.org/}, formando parte de la familia de herramientas de pruebas xUnit.
Es una de las librerías Java más utilizadas en proyectos de código abierto \cite{website:githubTOP}.

La versión 4 de JUnit extiende y simplifica la funcionalidad de anteriores versiones, haciendo un uso extensivo del sistema de anotaciones de Java.

\subsection{Apache Maven}
Maven es una herramienta de administración y construcción de proyectos usada principalmente con Java\footnote{http://maven.apache.org/}.
Similar en ciertos aspectos a Apache Ant, pero con un modelo de configuración más simple, y con el objetivo establecer una serie de `buenas prácticas' mediante sus configuraciones por defecto (`Convention over configuration').

Su funcionamiento se basa en la existencia de un fichero de configuración XML, el POM (Project Object Model).
Este fichero define la construcción, emisión documentación, empaquetado, pruebas, manejo de dependencias y múltiples otras tareas de manera centralizada \cite{mvnEx}.

Maven se estructura en torno a un núcleo central de tamaño mínimo, que puede ser extendido mediante \emph{plugins}.
Este núcleo es el encargado de descargar los \emph{plugins} que añaden la funcionalidad pedida desde sus repositorios remotos.

\subsection{SLF4J}
SLF4J (Simple Logging Facade for Java) sirve como interfaz para varias herramientas de logging\footnote{http://www.slf4j.org/}, permitiendo al usuario cambiar entre una y otra sin realizar modificaciones en el código.
Se establece como única dependencia obligatoria, encargándose por si misma de buscar una implementación compatible~\cite{website:slf4j}.

Esta herramienta permite trabajar con, por ejemplo, java.util.logging\footnote{http://docs.oracle.com/javase/8/docs/api/java/util/logging/package-summary.html}, logback\footnote{http://logback.qos.ch/} o log4j\footnote{http://logging.apache.org/log4j/1.2/}.

Es una de las librerías más utilizadas en proyectos de código abierto \cite{website:githubTOP}.

\subsection{Logback}
Logback es una herramienta de logging para Java que se establece como sucesor de Log4J.
Utiliza el API de SLF4J, por lo cuál puede ser libremente intercambiado por cualquier otro módulo de logging compatible.

La configuración de Logback se realiza mediante un pequeño fichero XML, en el cual se especifican los canales de salida, sean archivos o consola.
Siguiendo la estructura de Apache Maven, Logback permite establecer ficheros de configuración distintos para las etapas de desarrollo y para el producto final.
Logback refresca la configuración en tiempo de ejecución, permitiendo realizar cambios `en caliente' \cite{website:logback}.

Es la herramienta de logging más utilizada en proyectos de código abierto \cite{website:githubTOP}.

\subsection{JavaCC}
JavaCC\footnote{https://javacc.java.net/} (Java Compiler Compiler) es un generador de analizadores sintácticos Java.
Funciona mediante la especificación de una gramática en un formato propio.
A partir de este fichero, la herramienta genera un programa Java capaz de procesar texto y reconocer coincidencias con la gramática.

Además de generador en sí, JavaCC proporciona una serie de herramientas relacionadas, como un constructor de árboles (JJTree) \cite{website:javacc}.

\subsection{JGraphX}
JGraphX\footnote{https://github.com/jgraph/jgraphx} es una biblioteca de visualización de gráficos para Java de código abierto, derivada de la implementación comercial en JavaScript (mxGraph).
En las versiones anteriores la herramienta se conocía como JGraph \cite{website:jgraphx}.

Implementa una serie de funcionalidades básicas para trabajar con grafos, incluyendo visualización e interacción con los mismos.
De cara a aplicaciones Java, es compatible con Swing y permite exportar las imágenes generadas.

\subsection{Graphviz}
Graphviz es una herramienta de visualización de gráficos de código abierto\footnote{http://www.graphviz.org/}.
Proporciona una manera de representar información estructurada como diagramas, grafos y redes.
Sus aplicaciones se encuentran en arquitectura de redes, bioinformática, ingenieria del software, diseño de bases de datos y web, machine learning y diseño de interfaces gráficas \cite{website:graphviz}.

La herramienta toma descripciones de gráficos en texto plano y los transforma en diagramas, con varios formatos disponibles.
El formato `dot', en concreto, permite dibujar grafos dirigidos.

\subsection{Git}
Git es un sistema de control de versiones distribuido\footnote{http://git-scm.com/}, diseñado para trabajar con proyectos de cualquier tamaño.
Es un proyecto libre y de código abierto, licenciado bajo la GPL\footnote{http://www.gnu.org/copyleft/gpl.html}.

Una de las características distintivas de Git es su modelo de ramas.
Git permite mantener múltiples ramas locales independientes, cuya creación, modificación y combinado resulta poco costoso.
Esto permite aislar tareas, trabajando e incorporando cada una de manera totalmente separada.

Git es un sistema distribuido.
Esto quiere decir que el repositorio local mantiene una copia completa del repositorio remoto.
Esto permite una variedad de métodos de trabajo, y elimina puntos débiles del sistema \cite{website:git}.

\subsubsection{Github}
Github\footnote{https://github.com/} proporciona un servicio de almacenamiento de repositorios remotos y un entorno de colaboración para desarrolladores.
Actualmente cuenta con más de seis millones de usuarios y 13.5 millones de repositorios \cite{website:github}.

Github proporciona herramientas de seguimiento de proyectos, incluyendo una wiki y un issue tracker por repositorio.
También es compatible con otras aplicaciones web, como por ejemplo Pivotal Tracker, que facilitan su integración en el proceso de desarrollo.

\subsection{Pivotal tracker}
Pivotal Tracker\footnote{https://www.pivotaltracker.com} es una herramienta web para el manejo de proyectos ágiles, permitiendo la colaboración del equipo y la administración de tareas.
Su API permite la integración con otras herramientas de desarrollo, como por ejemplo Github.

Pivotal Tracker cuenta con una serie de herramientas de generación de informe, con las que podemos ver las tareas finalizadas en cada iteración, gráficos de \emph{burn-down}, etc.

\subsection{Moodle}
Moodle (\emph{Modular Object-Oriented Dynamic Learning Environment}) es una plataforma de enseñanza virtual (\emph{e-learning}) desarrollada como software libre.
Disponible en mas de 100 idiomas, Moodle habilita mas de 60 mil páginas de aprendizaje, con más de siete millones de cursos, 70 millones de alumnos y un millón de profesores \cite{website:moodle}.

Una de las características de Moodle es su capacidad para importar cuestionarios a partir de varios formatos, incluyendo texto plano, el formato propietario Gift y a partir de documentos XML.

\subsection{\LaTeX{}}
\LaTeX{} es un lenguaje de marcadores para la preparación de documentos comúnmente usado en publicaciones técnicas o científicas.

Es un procesador de texto «\emph{What you see is not what you get}», lo cual significa que lo que vemos durante la edición no es el 
documento final, sino las instrucciones para generarlo.
Esto permite separar casi completamente el contenido del documento de su formato.

\subsection{XML}
\emph{XML} (\emph{Extensive Markup Language}) es un lenguaje de etiquetas que se utiliza para crear documentos estructurados \cite{website:xml}, compuestos de entidades que contienen en su interior datos u otras entidades.
El estandar fue producido y es desarrollado por el \emph{World Wide Web Consortium}\footnote{http://www.w3.org/}.

Podemos verificar que un documento \emph{XML} tiene el formato correcto validandolo contra su esquema o \emph{DTD} (\emph{Document Type Definition}).
También podemos verificar que un documento está `bien formado', es decir, que cumple una serie de reglas gramaticales mínimas.

\subsection{ObjectAid UML Explorer}
ObjectAid UML Explorer\footnote{http://www.objectaid.com/} es un \emph{plugin} de visualización de código para eclipse.
Permite mostrar el código fuente de un proyecto Java en forma de diagramas UML, reflejando el estado y las relaciones en el código, y actualizandose a medida que el código cambia.
%\capitulo{4}{Aspectos relevantes del desarrollo del proyecto}

\section{Desarrollo de prototipo}
En el intervalo de tiempo situado entre la selección del proyecto y el comienzo formal de la primera iteración de desarrollo, se dedicaron recursos al desarrollo de un prototipo de la aplicación.
Los objetivos de esta fase eran la construcción de una interfaz gráfica que aproximase el diseño final (aunque no todas las funciones estuvieran implementadas), y el desarrollo de una funcionalidad parcial de la aplicación.
En este caso la funcionalidad parcial era la resolución del primer tipo de cuestiones.

\imagen{prototipo_interfaz}{Interfaz gráfica del prototipo de la aplicación}

El prototipo está limitado a trabajar con un único problema a la vez, las opciones de generación no son funcionales y el botón `Generar' produce siempre la misma expresión.
A pesar de estas limitaciones, el prototipo aproxima con bastante precisión la interfaz final, y forma la base de resolución de problemas de manera eficaz.
Es capaz, en concreto, de procesar una expresion regular cualquiera proporcionada como texto, construir a partir de ella un árbol sintáctico, aplicarle el algoritmo de Aho-Sethi-Ullman y mostrar los resultados obtenidos.

\section{Metodología ágil - Scrum}
El desarrollo del proyecto ha sido enmarcado dentro de una metodología ágil, Scrum\footnote{\url{https://www.scrum.org/}}.
Sin embargo, y debido a las diferentes condiciones entre equipo de desarrollo y las condiciones entre este proyecto y uno más habitual, se han realizado ciertas modificaciones.

En cuanto a roles, volcamos los tres roles que define Scrum («\emph{Product Owner}», «\emph{Development Team}» y «\emph{Scrum Master}») sobre el desarrollador, dado que no tenemos un equipo como tal.
El rol de «\emph{Product Owner}», especialmente la definición de tareas para el \emph{backlog}, se realizará conjuntamente con el tutor.

Se definen \emph{sprints} de dos semanas, entre los cuales se realiza una reunión con el tutor del proyecto, y al final de cada cuál se completa un entregable.
Esta reunión cumple las funciones de «\emph{Sprint Planning Meeting}» y «\emph{Sprint Review}», y en ella se revisan los resultados del \emph{sprint} que termina y se definen los objetivos del que comienza.
No se realizan reuniones diarias («\emph{Daily Scrum}»), resultando innecesarias al contar con un equipo de desarrollo de una persona.

El seguimiento de los sprint se realiza mediante el uso de una herramienta online, Pivotal Tracker\footnote{\url{https://www.pivotaltracker.com/s/projects/1026880}}, de manera que resulta accesible a tutor y desarrollador.
La integración de esta herramienta con el sistema de control de versiones utilizado, Github\footnote{\url{https://github.com/RobertoIA/PLQuiz}}, permite la terminación de tareas directamente desde el \emph{commit} que las finaliza.
Se consideran validadas las tareas cuando se completa su documentación (en forma de \emph{javadoc}), se escriben sus pruebas y estas se ejecutan correctamente.

En el apéndice C describimos en detalle el desarrollo de cada \emph{sprint}, asi como su duración, objetivos y resultados.

\section{Estructura de paquetes}
La filosofía seguida a la hora de diseñar la estructura de la aplicación se centra en los conceptos de modularidad y reusabilidad.
Cada paquete pretende ser independiente, y las dependencias externas se reducen lo más posible.
Esto se traduce en un uso frecuente del patrón `fachada'.

Cada paquete principal tiene una estructura de dos niveles: el primer nivel contiene las clases `fachada' --- visibles al resto de la aplicación ---, mientras que el segundo (\emph{.datos}) contiene la lógica y estructuras de datos internas --- utilizadas solo por el paquete `padre' ---.

Podemos dividir los paquetes principales de la aplicación en tres clases:
\begin{itemize}
	\item Paquete de interfaz gráfica (\emph{es.ubu.inf.tfg.ui}).
	Contiene las clases que definen la apariencia y comportamiento de la interfaz de usuario.
	Es, por pura necesidad, el paquete con más dependencias externas, dependiendo tanto del paquete que define la representación de documentos como de los paquetes que definen cada clase de problemas.
	\item Paquete de construcción de documentos (\emph{es.ubu.inf.tfg.doc}).
	Dependen, por necesidad, de los paquetes que definen las clases de problemas que van a representar.
	\item Paquetes de problemas (\emph{es.ubu.inf.tfg.regex.thompson} y \emph{es.ubu.inf.tfg.regex.asu}).
	Contienen las clases encargadas de representar las estructuras internas del problema, sus soluciones, y aquellas necesarias para generarlos.
	Dependen únicamente del paquete de representación de expresiones regulares, y son independientes el uno del otro.
	\item Paquete de representación y procesamiento de expresiones regulares (\emph{es.ubu.inf.tfg.regex.parser} y \emph{es.ubu.inf.tfg.regex.datos}).
	Contienen las clases necesarias para procesar expresiones regulares a partir de cadenas de caracteres, representarlas y operar sobre ellas.
	Son la base sobre la cual trabajan los problemas.
\end{itemize}

\imagen{diagrama_paquetes}{Diagrama de paquetes y sus dependencias}

Como podemos ver, los paquetes encargados del procesamiento de expresiones regulares son la base del problema.
Junto con los paquetes de problemas forman el núcleo de la aplicación, y pueden ser utilizados de manera independiente, al carecer de dependencias externas.

Sobre estos paquetes centrales se aposentan los paquetes de representación de documentos, que toman los datos de los documentos y les dan un formato adecuado.
En un nivel superior esta la interfaz de usuario, que toma el conjunto de datos obtenido del resto de paquetes y lo presenta al usuario.

Esta estructura en niveles significa que podemos tomar cualquier paquete, y utilizarlo de manera completamente independiente, siempre que dispongamos de aquellos que se encuentran en niveles inferiores.

\section{Java 8 - \emph{expresiones lambda} y \emph{streams}}
Dos de las novedades introducidas con la versión 8 de Java son los \emph{streams}\footnote{\url{http://docs.oracle.com/javase/8/docs/api/java/util/stream/package-summary.html}} y las \emph{expresiones lambda}\footnote{\url{http://docs.oracle.com/javase/tutorial/java/javaOO/lambdaexpressions.html}}.

Los \emph{streams} permiten manipular colecciones, aplicándoles operaciones típicas de programación funcional, como pueden ser \emph{map} o \emph{reduce}.
Una de las ventajas de estas operaciones son que permiten la aplicación de operaciones a cada elemento de una colección sin iterar explicitamente, o la obtención de subconjuntos de la misma que cumplan una condición cualquiera.

La aplicación de operaciones se realiza mediante \emph{expresiones lambda}.
Las \emph{expresiones lambda} permiten definir una clase de un solo método de manera compacta, encapsulado la operación.

Dado que la salida de la nueva versión de Java ocurrió cuando el desarrollo de la aplicación ya estaba avanzado, la utilización de las nuevas herramientas se da de manera ocasional, y tiene un impacto reducido.

El principal uso de los \emph{stream} se encuentra en las clases encargadas de la generación de expresiones regulares.
Por ejemplo, las operaciones sobre `poblaciones' de expresiones regulares dentro de un algoritmo genético resultan simples si disponemos de medios de `filtrar' subconjuntos dentro de la colección.
Usando la función de evaluación del algoritmo genético como argumento para la operación de filtrado sobre el \emph{stream} permite obtener el conjunto de la población que queremos preservar, mutar o cruzar de manera simple y sin recurrir a la iteración explicita sobre la colección.
%\capitulo{5}{Trabajos relacionados}
%\capitulo{6}{Conclusiones y líneas de trabajo futuras}

\section{Algoritmos de generación de árboles de expresión regular}
El sistema de generación de expresiones regulares actual implementa el método «full» descrito por John R. Koza \cite{koza92}.
Dicho método genera árboles en los que todas las hojas tienen la misma altura.
\\
Este método permite generar expresiones de forma rápida y resulta relativamente simple de implementar, dado que los árboles no varían en profundidad.
\\
Como desventaja, el método «full» genera un rango de expresiones limitado.
Esta característica supone un problema a la hora de utilizar las expresiones para generar problemas.
\\
Si agrupamos los problemas según los valores de las características que encontramos para generarlos, podemos ver que la distribución de los mismos no resulta uniforme, sino que se generan muchos problemas con ciertos valores y muy pocos o ninguno con otros.
Esto resulta problemático si presentamos una interfaz de usuario que permita introducir valores arbitrarios.
\\
En la literatura encontramos definidos otros dos posibles algoritmos, «grow» y «half-and-half» \cite{koza92}.
La implementación de uno de estos algoritmos (o de algún otro equivalente) supondría una mejora del sistema de generación y permitiría comprobar como de equilibrada está la distribución de resultados.
\\
Las mejoras en el algoritmo de generación de árboles afectarían también a los algoritmos de búsqueda, permitiendo tal vez utilizar métodos más rápidos o eficientes, y generar problemas más complejos.

\appendix

\apendice{Generación de expresiones regulares}

Trabajar con una expresión regular dentro de la aplicación implica traducirla a una estructura de datos adecuada.
Contando con que la expresión es introducida como una cadena de caracteres, utilizaremos un parser para convertirla en un árbol binario.
Los algoritmos de generación de árboles son un tema muy estudiado dentro de la programación genética, lo cuál nos permite generar la expresión directamente en forma de árbol implementando un algoritmo ya conocido.
\\
El algoritmo concreto implementado es el del método "full" \cite{koza92}.
Este método involucra la creación de árboles cuya profundidad viene definida por la longitud del camino entre un extremo cualquiera y la raíz.
Esto quiere decir que generaremos árboles binarios en los que cada cada nodo no hoja tiene 1 o 2 hijos, y en el que todas las hojas se encuentran a la misma profundidad.
\\
El algoritmo funciona restringiendo la selección de etiquetas para los nodos generados en función de su profundidad.
Un nodo símbolo, por ejemplo, solo estará permitido en la profundidad máxima, mientras que un nodo cerradura estará permitido en cualquier profundidad excepto la máxima.
Tomaremos consideraciones adicionales a la hora de generar los hijos de un nodo, ya que el número de los mismos variará en función de la operación que contenga el padre.
Un nodo cerradura, por ejemplo, tiene un solo hijo, mientras que un nodo concatenación tendrá dos.

\section{Generación de problemas}

A la hora de generar problemas tomamos tres criterios:
\begin{itemize}
	\item El número de símbolos presentes en la expresión regular (sin contar el vacío).
	\item El número de estados en su función de transición.
	\item La presencia o ausencia del símbolo vacío.
\end{itemize}
Tanto el número de símbolos como la presencia del símbolo vacío dependen directamente de la expresión regular a partir de la cual construimos el problema.
El número de estados de la función de transición será el mismo para cualquier problema de un mismo tipo que comparta la misma expresión regular.
Podemos decir, por lo tanto, que las características que buscamos en un problema dependen exclusivamente de la expresión regular con la cuál lo construimos.
\\
La generación de problemas por lo tanto consistirá en generar expresiones regulares cuyo problema asociado cumpla las características pedidas.
Siendo el objetivo la generación de problemas con unas características dadas, deben determinarse los algoritmos de búsqueda a utilizar y los parámetros óptimos para los mismos.

\section{Coste de generación}

Dado que los árboles de expresión son árboles binarios, sabemos que el número máximo de nodos está acotado superiormente por la n
\[
expresi\acute{o}n\acute{u}mero \; de \; nodos \; \leq \; 2^{\;profundidad} - 1
\]
La generación de una expresión regular crea los nodos del árbol de la expresión uno a uno, por lo tanto el coste de generar una expresión crecerá con el número de nodos de la misma.
\\
Para comprobar estos datos de manera experimental obtendremos los tiempos medios de generación de 100.000 expresiones regulares para cada profundidad entre 0 y 8, y comparamos su crecimiento con el número máximo de nodos asociado a cada profundidad.

\imagen{profundidad}{Tiempo de generación de expresiones y número de nodos}

Podemos ver como el tiempo de generación de una expresión regular depende del número de nodos que contenta y, por lo tanto, de la profundidad.
Por lo tanto será preferible generar expresiones lo menos profundas posible, en concreto limitándonos a una profundidad máxima de 6.
\\
Destacamos que los valores del tiempo dependen de la máquina en que se realizan las pruebas.
Este análisis pretende analizar el crecimiento de la función tiempo, no sus valores exactos.

\section{Regiones aceptables}

Consideramos expresiones 'útiles' aquellas que contienen entre 2 y 6 símbolos distintos (sin incluir el vacío), y entre 3 y 15 estados distintos al generar a partir de la expresión un problema de tipo Aho-Sethi-Ullman o de construcción de subconjuntos.
Estos rangos corresponden con los problemas que consideramos apropiados para resolver de manera manual, y con lo permitido por la interfaz.
El generador es capaz de trabajar con rangos arbitrarios, pero los problemas que se forman con expresiones fuera de estos rangos son o demasiado complejos para resolver de manera manual, o completamente triviales.
\\
El conocimiento de que profundidades de árbol corresponden con más expresiones generadas dentro de los rangos deseados nos permite establecer los límites óptimos para utilizar con los generadores de problemas.
\\
Este análisis pretende determinar cuantas de las expresiones regulares generadas aleatoriamente encajan dentro de los rangos deseados, y en que medida depende de la profundidad del árbol sintáctico.
La intuición nos dice que los árboles más profundos corresponderán con las expresiones más complejas (más símbolos y/o estados).

\subsection{Aho-Sethi-Ullman}

Para realizar las pruebas generaremos 100.000 expresiones regulares conteniendo el símbolo vacío, y otras tantas sin contenerlo, para cada profundidad de árbol de entre 1 y 6.
Con cada una de las expresiones construiremos un problema de tipo Aho-Sethi-Ullman y comprobaremos cuantas de las expresiones generadas entran dentro de los rangos de símbolos y estados que consideramos aceptables.

\imagen{prof-ASU-NV}{Aho-Sethi-Ullman (no vacío) aceptables según profundidad}
\imagen{prof-ASU-V}{Aho-Sethi-Ullman (vacío) aceptables según profundidad}

Los resultados muestran que los árboles de profundidad 4 generan el mayor número de expresiones en la zona deseada.
También podemos ver claramente que las expresiones con árbol sintáctico de profundidad entre 3 y 6 generan una mayoría de expresiones regulares aceptables, tanto si incluimos el símbolo vacío como si no.
\\
Por lo tanto el método de búsqueda para problemas de Aho-Sethi-Ullman utilizará expresiones de profundidad entre 3 y 6.

\subsection{Construcción de subconjuntos}

Para el análisis de problemas de construcción de subconjuntos repetimos el mismo proceso que en el apartado anterior, generando expresiones regulares para cada combinación de tipo y profundidad, y construyendo problemas de construcción de subconjuntos con ellas.

\imagen{prof-CS-NV}{Construcción de subconjuntos (no vacío) aceptables según profundidad}
\imagen{prof-CS-V}{Construcción de subconjuntos (vacío) aceptables según profundidad}

Según los resultados podemos ver que las expresiones dentro del rango tienen mayoritariamente profundidades de entre 1 y 5, y que las expresiones de profundidad 6 están completamente fuera del rango.
\\
Por lo tanto el método de búsqueda para problemas de construcción de subconjuntos utilizará expresiones de profundidad entre 1 y 5.

\section{Distribución de resultados aceptables}

Partiendo de los resultados que encontramos dentro de la región aceptable, es importante identificar como se distribuyen esos resultados.
Asumimos que no todas las combinaciones de símbolo y estado van a aparecer con la misma frecuencia dentro de la región aceptable.
\\
Dado que la frecuencia de aparición de expresiones en la región aceptable varía según la profundidad, examinaremos cada profundidad utilizada por separado.
De esta manera podremos determinar si ciertas combinaciones de símbolo y estado son más probables en ciertas profundidades.
\\
Los datos experimentales utilizados son los mismos que en los apartados anteriores.

\subsection{Aho-Sethi-Ullman}

Agrupando los datos según profundidades podemos ver nuevamente las agrupaciones según profundidad.
Más interesante resulta que las expresiones generadas se agrupan, independientemente de la profundidad, alrededor de los mismos puntos.
\\
Es interesante notar que, aunque las agrupaciones no depende de la profundidad, las profundidades mayores parecen agruparse mas hacia la izquierda.
Esto implicaría que las profundidades mayores generan problemas con menos estados y símbolos o, más probablemente, que nuestro rango aceptable es demasiado restringido para ver la perspectiva completa.

\imagen{dist-ASU-NV}{Distribución de Aho-Sethi-Ullman (no vacío)}
\imagen{dist-ASU-V}{Distribución de Aho-Sethi-Ullman (vacío)}

Vemos que el comportamiento es similar entre expresiones que contienen el elemento vacío y aquellas que no.
\\
Esto indica, de manera bastante clara, que ciertas combinaciones de número de símbolos y estados son mucho más probables.
O, por otra parte, que en las profundidades con las que estamos trabajando tienden a generarse expresiones con unas características especificas.
\\
Por ejemplo, es posible que la combinación de 3 símbolos y 8 a 14 estados se dé con alta probabilidad en profundidades de 6 o mayores, o que necesite expresiones regulares con árboles que tengan hojas a distintas profundidades.

\subsection{Construcción de subconjuntos}

En los problemas de construcción vemos los mismos problemas que en los de Aho-Sethi-Ullman.
Las expresiones regulares se agrupan en torno a unas ciertas combinaciones de número de símbolos y estados, aunque en este caso las combinaciones son diferentes que en el apartado anterior.

\imagen{dist-CS-NV}{Distribución de construcción de subconjuntos (no vacío)}
\imagen{dist-CS-V}{Distribución de construcción de subconjuntos (vacío)}

Es interesante remarcar que las agrupaciones parecen ampliarse hacia la derecha.
Es decir, cuanto más elevado el número de símbolos, más variedad de estados generan los problemas.

\section{Algoritmos de búsqueda}

Una vez definidos los parámetros con los cuales vamos a generar las expresiones, queda pendiente encontrar un algoritmo de búsqueda que nos permita encontrar los problemas dados.
\\
Es probable que dos problemas generados a partir de expresiones regulares tengan características similares, pero no tenemos ninguna garantía al respecto.
Es decir, es posible que dos expresiones vecinas tengan características bastante diferentes.
Descartamos, por lo tanto, algoritmos de búsqueda como el recocido simulado, que se desplazan por el espacio de búsqueda de vecino a vecino.

\subsection{Búsqueda aleatoria}

El algoritmo de búsqueda aleatoria consiste en generar soluciones completamente al azar hasta encontrar la correcta.
Lo tomamos como posibilidad dado que resulta muy sencillo de implementar, y de que resultará eficiente en el caso de búsquedas sencillas.

% ASU, CS

\subsection{Algoritmo genético}

% ASU, CS, Reproducción o mutación

\section{Conclusiones}

% puede ser que la forma de los árboles de expresiones (nodos hoja con misma profundidad) favorezcan ciertas combinaciones de simbolos / estados y desfavorezcan otras.

\bibliographystyle{plain}
\bibliography{PLQuiz}

\end{document}