\capitulo{3}{Técnicas y herramientas}

\section{Técnicas}

\section{Herramientas}

Las herramientas utilizadas durante el desarrollo del proyecto son las siguientes:

\subsection{Java}

Java es un lenguaje de programación de propósito general, concurrente, basado en clases y orientado a objetos.
Se trata de un lenguaje fuerte y estáticamente tipado, y distingue claramente entre errores en tiempo de compilación y errores en tiempo de ejecución.
\\
La compilación de un programa Java implica su traducción a representación 'byte code', que es independiente de la plataforma.
Esta representación intermedia pasa a ejecutarse sobre la máquina virtual Java.
Por lo tanto, el código Java sigue la filosofía 'write once, run anywhere', siendo compatible con cualquier arquitectura que disponga de una máquina virtual compatible.
\\
Java es un lenguaje de relativo alto nivel, en el sentido de que los detalles de representación no son accesibles al programador.
Incluye una administración automática del almacenamiento de datos en forma del recolector de basura, diseñado para evitar los problemas relacionados con la liberación manual de memoria.
\cite{jls8}
El lenguaje Java es, a fecha de 2014, uno de los lenguajes de programación más populares del mundo \cite{tiobe}.

\subsubsection{Java 8}

Java SE 8 es la edición más reciente del lenguaje de programación Java, y representa la mayor evolución del mismo en su historia.
\\
Añade a la especificación un conjunto relativamente pequeño de nuevas características, mediante las cuales permite combinar modelos de programación orientada a objetos y funcional.
Estos cambios pretenden favorecer una serie de buenas prácticas - inmutabilidad, ausencia de estado, composición - a la vez que mantienen las características de Java - legibilidad, simplicidad y universalidad.
\\
Las librerías de la plataforma Java mantienen una evolución paralela a la del lenguaje.
Esto significa que el uso de las nuevas características - expresiones lambda, referencias a métodos, interfaces funcionales - es directamente compatible.
\cite{jls8}

\subsubsection{Eclipse}

Eclipse es un entorno de desarrollo integrado (IDE) escrito principalmente en Java y compatible con una amplia variedad de lenguajes.

