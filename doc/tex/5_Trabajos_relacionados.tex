\capitulo{5}{Trabajos relacionados}

\section{BURGRAM --- Software de generación y simulación de tablas de análisis sintáctico}
BURGRAM\footnote{Carlos Gómez Palacios, Enero 2008, Universidad de Burgos} es una herramienta que permite seguir paso a paso el proceso de los distintos modos de análisis sintáctico (tanto ascendente como descendente).
La aplicación permite editar gramáticas, visualizar el proceso completo de operación del algoritmo sobre las mismas, y la generación de informes con los resultados.

\section{Generación de preguntas para la docencia on-line de Estructuras de Datos}
Este proyecto\footnote{Jesús Javier Rodríguez Terrados y Pablo Dobarco García, Julio 2013, Universidad de Burgos} presenta una aplicación capaz de generar preguntas relacionadas con los temas dados en la asignatura de `Estructuras de datos'.
Las cuestiones se generan en formatos \emph{GIFT} y XML, ambos compatibles con la plataforma de aprendizaje virtual Moodle.

\section{THOTH --- Software de apoyo a la enseñanza de autómatas finitos, lenguajes regulares y gramáticas}
THOTH\footnote{Andrés Arnáiz Moreno y Álvar Arnáiz González, Julio 2006, Universidad de Burgos} es un proyecto compuesto de una aplicación de escritorio y una página web, ambos accediendo a la misma funcionalidad.
La integración de ambas partes resulta posible gracias a Java, escribiendo la aplicación de escritorio en \emph{Swing} y utilizando \emph{applets} para insertar la funcionalidad en la página web.

Permite el manejo de autómatas finitos y lenguajes formales, con la idea de proveer una serie de ejemplos interactivos que complementen los conceptos teóricos de la asignatura `Autómatas y Lenguajes Formales'.

\section{Regular Expression to NFA}
Esta aplicación web\footnote{\url{http://hackingoff.com/compilers/regular-expression-to-nfa-dfa}} permite procesar expresiones regulares y aplicar sobre ellas el algoritmo de McNaughton-Yamada-Thompson, generando el autómata finito no determinista correspondiente.
A continuación aplica al resultado el algoritmo de construcción de subconjuntos, construyendo el autómata finito determinista equivalente al anterior.

Permite, por lo tanto, visualizar el proceso de construcción de los distintos tipos de autómata partiendo de una expresión regular, y las diferencias entre ambos.