\capitulo{2}{Conceptos teóricos}

\section{Algoritmo de McNaughton-Yamada-Thompson}

Utilizaremos el algoritmo de McNaughton-Yamada-Thompson \cite{dragon} \cite{Thompson:1968:PTR:363347.363387} para convertir una expresión regular cualquiera en un autómata finito determinista que defina el mismo lenguaje.

El algoritmo construye un autómata finito no determinista (AFND) a partir de una expresión regular analizando de forma recursiva sus partes constituyentes (subexpresiones).
El algoritmo crea un AFND para cada subexpresión y las une teniendo en cuenta el operador que las combina
El algoritmo puede implementarse mediante un recorrido de arriba a abajo del árbol sintáctico que representa la expresión regular.

Para empezar realizamos un análisis de la expresión regular, dividiendola en las expresiones que la componen.
Para construir el autómata final iremos construyendo autómatas para cada una de las subexpresiones, y uniendo los autómatas según las operaciones entre expresiones.

\subsection{Expresión vacía}
En el caso de que quisieramos generar un autómata para la expresión vacía:

\imagen{thompson-vacio}{Autómata para una expresión vacía}

Tenemos un estado de inicio, un estado final y una transición entre ambos.

\subsection{Expresión para cadena de un solo símbolo}
Tomamos una expresión cuyo lenguaje asociado solo reconoce cadenas de un símbolo, siendo el símbolo uno cualquiera.

\imagen{thompson-simbolo}{Autómata para un solo símbolo}

De nuevo encontramos un estado inicial, un estado de aceptación y una transición entre ambos.

Una vez tenemos los autómatas de las subexpresiones mínimas pasamos a unirlos en autómatas mayores.
La manera en que unimos subexpresiones depende de la operación que une las ramas del árbol donde se encuentran.

\subsection{Operación de unión}
Supongamos una operación de unión entre dos subexpresiones.
En la nueva expresión contamos con un nuevo estado inicial con transiciones vacías hacia los estados iniciales de las subexpresiones.
Vemos también un nuevo estado de aceptación hacia el que tienen transiciones los estados de aceptación de las subexpresiones.

\imagen{thompson-union}{Autómata para la operación de unión}

\subsection{Operación de concatenación}
Supongamos una operación de concatenación entre dos subexpresiones.
En primer lugar combinaremos el estado de aceptación de la primera con el estado de inicio de la segunda, dejando un único estado.

Tomamos como estado inicial el estado inicial de la primera, y como estado de aceptación el estado de aceptación de la segunda.

\imagen{thompson-concatenacion}{Autómata para la operación de concatenación}

\subsection{Operación de cierre}
El autómata más complejo se genera con las operaciones de cierre.
Tomamos nuevos estados de inicio y de aceptación, con una transición vacía del uno al otro.
Asimismo, tendremos una transición vacía entre los estados de aceptación e inicio de la subexpresión.
Como último paso, uniremos el nuevo estado de inicio con el de inicio de la subexpresión, y el de aceptación de la subexpresión con el nuevo estado de aceptación.

\imagen{thompson-cierre}{Autómata para la operación de cierre}

\subsection{Ejemplo}
Tomamos como ejemplo la expresión regular $(a|b)^*$, que define el lenguaje que contiene todas las palabras que contienen cualquier número de $a$ y $b$.

A partir de la expresión regular definimos el siguiente árbol

\imagen{thompson-ejemplo-arbol}{Árbol de la expresión regular $(a|b)^*$}

Una vez hemos definido el árbol comenzamos por construir los autómatas de las subexpresiones más pequeñas, que serán las correspondientes a las expresiones $(a)$ y $(b)$.
Las subexpresiones generan los siguientes autómatas separados

\imagen{thompson-ejemplo-simbolos}{Autómatas para las expresiones $(a)$ y $(b)$}

Consultando el árbol de la expresión, vemos que las dos subexpresiones cuelgan de un nodo unión, por lo que aplicamos las reglas ya conocidas, y obtenemos el siguiente autómata

\imagen{thompson-ejemplo-union}{Autómata para la expresión $a|b$}

Finalmente generamos el autómata completo utilizando el autómata de la subexpresión anterior y añadiendo alrededor los nodos correspondientes a la operación de cierre.
Con esto hemos conseguido el autómata capaz de reconocer el lenguaje definido por la expresión $(a|b)^*$.

\imagen{thompson-ejemplo-cierre}{Autómata para la expresión $(a|b)^*$}

\section{Algoritmo de Aho-Sethi-Ullman}
El algoritmo de Aho-Sethi-Ullman \cite{dragon} \cite{5221603} consiste en el cálculo de una serie de funciones a partir del árbol de la expresión regular, a partir de los cuales se genera la tabla de transición del autómata.

Las funciones calculadas son anulable, primera posición, ultima posición y siguiente posición.

\subsection{Función anulable}
La función anulable establece si un nodo dado del árbol sintáctico incluye el símbolo vacío $ \epsilon $ en su lenguaje.
Es decir, define el hecho de que el lenguaje definido por una subexpresión pueda incluir la cadena vacía.

Las hojas $ \epsilon $ y las operaciones de cierre son siempre anulables.
Las operaciones de unión son anulables si uno de sus nodos hijo lo es, y las de concatenación si ambos nodos hijo son anulables.

\subsection{Función de primera posición}
La primera posición (o \emph{primerapos}) de un nodo se compone del conjunto de posiciones del subárbol que corresponden con los primeros símbolos posibles para las cadenas que define.
Es decir, los primeros símbolos posibles para las cadenas de la expresión.

La primera posición de un nodo símbolo es su posición, a menos que el símbolo sea $ \epsilon $, en cuyo caso su función \emph{primerapos} es el conjunto vacío.
Los nodos cierre heredan la \emph{primerapos} de su nodo hijo.

En el caso de los nodos unión tenemos que la \emph{primerapos} del nodo es la unión de las funciones de sus nodos hijo.
Los nodos concatenación toman la \emph{primerapos} de su hijo izquierdo, y si éste es anulable añaden la de su hijo derecho.

\subsection{Función de última posición}
La función de última posición (o \emph{últimapos}) corresponde con el conjunto de posiciones del subárbol que pueden estar en última posición de alguna de las cadenas posibles.
Es el opuesto de la función \emph{primerapos}.

Al igual que en la función \emph{primerapos}, los conjuntos \emph{últimapos} de los nodos hoja son la posición del nodo, o el conjunto vacío en los nodos $ \epsilon $.
Los nodos cierre también se comportan igual, heredando la \emph{últimapos} de su nodo hijo.

En los nodos unión tenemos como \emph{últimapos} la unión de los conjuntos \emph{últimapos} de sus nodos hijo.
Los nodos concatenación toman el conjunto \emph{últimapos} de su nodo hijo derecho, y si éste es anulable añaden la \emph{últimapos} de su nodo hijo izquierdo.

\subsection{Función de siguiente posición}
La función siguiente posición (o \emph{siguientepos}) nos permite ver que posición o posiciones siguen a una posición dada en una expresión.
Existen dos posibilidades:
\begin{itemize}
	\item Si tenemos un nodo concatenación, las posiciones del conjunto \emph{primerapos} del nodo hijo derecho esta en el conjunto \emph{siguientepos} de cada posición en el conjunto \emph{últimapos} del nodo hijo izquierdo.
	\item Si tenemos un nodo cierre, todas las posiciones del conjunto \emph{primerapos} están en el \emph{siguientepos} de cada elemento del conjunto \emph{últimapos}.
\end{itemize}

\subsection{Tablas de transición}
Una tabla de transición es una de las posibles representaciones de un autómata finito, con las filas representando los estados y las columnas representando las entradas.

Cada entrada de la tabla consiste en el estado o estados a los que pasa un autómata dada una combinación inicial de estado y entrada.
Tiene la ventaja de que resulta sencillo encontrar transiciones concretas, pero ocupa demasiado espacio cuando trabajamos con autómatas de muchos estados o con un número alto de posibles entradas.
