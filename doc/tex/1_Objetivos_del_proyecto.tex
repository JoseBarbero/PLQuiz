\capitulo{1}{Objetivos del proyecto}

\section{Objetivos técnicos}
Los objetivos del proyecto, vistos desde el punto de vista del desarrollo técnico de la aplicación, pueden desglosarse en los siguientes puntos:
\begin{itemize}
	\item Desarrollo de una interfaz gráfica multiplataforma.
	\item Procesado y manipulación de expresiones regulares mediante el uso de un procesador de lenguajes, construido en torno a la herramienta \emph{JavaCC}.
	\item Resolución de problemas de aplicación del algoritmo de Aho-Sethi-Ullman a partir de las expresiones regulares procesadas.
	\item Resolución de problemas de aplicación del algoritmo de McNaughton-Yamada-Thompson a partir de las expresiones regulares procesadas.
	\item Agregación de los problemas en cuestionarios.
	\item Exportado de cuestionarios a dos formatos:
	\begin{itemize}
		\item Formato \LaTeX{}, generando cuestionarios listos para impresión, con intención de ser usados como examenes en papel.
		\item Formato \emph{XML} compatible con la plataforma de aprendizaje virtual Moodle.
	\end{itemize}
	\item Elaboración de un documento detallando los aspectos relevantes del desarrollo y funcionamiento de la aplicación.
\end{itemize}

Sin embargo los objetivos no han permanecido estáticos durante el periodo de desarrollo.
A los puntos iniciales se han añadido los siguientes:
\begin{itemize}
	\item Generación de problemas aleatorios, utilizando algoritmos de programación genética y métodos de búsqueda.
	\item Generación de problemas en bloque, produciendo cuestionarios completos con una variedad de problemas distintos.
	\item Exportado de imágenes en múltiples formatos, incluyendo exportado directo, exportado de programas \emph{Graphviz} y codificación en base 64.
	\item Exportado de cuestionarios a un formato adicional, HTML, permitiendo la vista previa de cuestionarios dentro de la interfaz gráfica.
	\item Implantado de un sistema de `log', facilitando el diagnóstico de problemas durante el desarrollo de la aplicación.
\end{itemize}

\section{Objetivos académicos}
Además de los objetivos relacionados directamente con el desarrollo del proyecto concreto, se consideran otros más generales, particularmente los detallados en la guía docente.
De estos destaco especialmente los siguientes:
\begin{itemize}
	\item Aplicación de los conocimientos teóricos y prácticos adquiridos en al asignatura al desarrollo de un proyecto de más envergadura y durante un periodo de tiempo considerable.
	\item Planteamiento y resolución de un problema real, con aplicaciones y usos concretos.
	\item Aprendizaje de nuevos temas relacionados con la titulación, trabajando con nuevos métodos y tecnologías y adaptando a ellos el modo de trabajo.
	\item Desarrollo de la capacidad de exposición y comunicación, especialmente durante la defensa, en el desarrollo de la memoria y en la preparación de la documentación de la aplicación.
\end{itemize}

\section{Objetivos personales}
En último lugar, la realización de este proyecto me ha dado la oportunidad de completar una serie de objetivos a nivel personal.
Destaco especialmente el aprendizaje de nuevas tecnologías y herramientas, que aunque ya me resultaban conocidas o familiares, no había tenido oportunidad de aplicar seriamente.

Entre las herramientas utilizadas en el proyecto, las utilizadas para el `logging', SLF4J y Logback, resultan mucho más útiles al ser aplicadas a bases de código grandes.
Su uso en el proyecto no solo ha facilitado el diagnóstico de errores y el control del flujo del programa, sino que me ha proporcionado conocimientos que probablemente pueda aplicar más adelante.

También destaco el uso de la versión 8 de Java, que apareció cuando el proyecto ya estaba avanzado.
Especialmente el añadido de los `stream', que permiten el manejo de datos de manera similar a la programación funcional, resultó de ayuda durante el desarrollo de ciertas partes de la aplicación.
Su uso en el proyecto me ha permitido ponerme al día de las novedades en Java, utilizándolas para resolver problemas reales.